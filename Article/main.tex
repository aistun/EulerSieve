\documentclass[a4paper]{easychair}

\usepackage[utf8]{inputenc}
\usepackage[french]{babel}
\usepackage{todonotes}
\usepackage{./why3lang}

% \pagestyle{empty} % pour la version finale EasyChair

\begin{document}

\title{...}
\titlerunning{...}
\author{...}
\authorrunning{...}
\institute{Université Paris-Saclay}
\maketitle

\begin{abstract}
  Dans cet article, ...
\end{abstract}

\section{Introduction}

\cite{taocp}

\section{La preuve}

let variable \whyf{x} est affichée correctement

le mot-clé \whyf{requires} aussi

\begin{why3}
  let f (x: int) : int
    requires { x > 0 }
  =
    x + 1
\end{why3}

\begin{ocaml}
  let f x = x + 1
\end{ocaml}

\section{Conclusion}
\label{sec:conclusion}

%%%%%%%%%%%%%%%%%%%%%%%%%%%%%%%%%%%%%%%%%%%%%%%%%%%%%%%%%%%%%%%%%%%%%%%%%%%%%%

\paragraph{Remerciements.} ...

%%%%%%%%%%%%%%%%%%%%%%%%%%%%%%%%%%%%%%%%%%%%%%%%%%%%%%%%%%%%%%%%%%%%%%%%%%%%%%
\bibliographystyle{plain}
\bibliography{./biblio}

\end{document}


% Local Variables:
% compile-command: "make"
% ispell-local-dictionary: "francais"
% End:
